%!TEX root=../main.tex
\ifx\allfiles\undefined
  \documentclass{article}
  \usepackage[a4paper,centering,scale=0.8]{geometry}
  \usepackage{newtxtext}
\usepackage{lipsum} % 该宏包是通过 \lipsum 命令生成一段本文,正式使用时不需要引用该宏包
\usepackage[dvipsnames,svgnames]{xcolor}
\usepackage[strict]{changepage} % 提供一个 adjustwidth 环境
\usepackage{framed} % 实现方框效果
\usepackage{tcolorbox}

% ------------------------------------------
\usepackage{amsfonts}
\usepackage{hyperref}
% -----------------------------------------
\usepackage{amsmath}
\usepackage{amsthm}
\newtheorem{define}{Definition}[section]
\newtheorem{theorem}{Theorem}[section]
% -----------------------------------------

\tcbuselibrary{most}
% environment derived from framed.sty: see leftbar environment definition
\definecolor{greenshade}{rgb}{0.90,0.99,0.91} % 文本框颜色
\definecolor{redshade}{rgb}{1.00,0.90,0.90}
\definecolor{yellowshade}{rgb}{0.90,0.90,1.0}
% ------------------******-------------------
% 注意行末需要把空格注释掉,不然画出来的方框会有空白竖线
\newenvironment{Define}{%
\def\FrameCommand{%
\hspace{1pt}%
{\color{LightCoral}\vrule width 2pt}%
{\color{redshade}\vrule width 4pt}%
\colorbox{redshade}%
}%
\MakeFramed{\advance\hsize-\width\FrameRestore}%
\noindent\hspace{-4.55pt}% disable indenting first paragraph
\begin{adjustwidth}{}{7pt}%
\vspace{2pt}\vspace{2pt}%
}
{%
\vspace{2pt}\end{adjustwidth}\endMakeFramed%
}
% ------------------******-------------------
\newenvironment{Proof}{%
\def\FrameCommand{%
\hspace{1pt}%
{\color{Green}\vrule width 2pt}%
{\color{greenshade}\vrule width 4pt}%
\colorbox{greenshade}%
}%
\MakeFramed{\advance\hsize-\width\FrameRestore}%
\noindent\hspace{-4.55pt}% disable indenting first paragraph
\begin{adjustwidth}{}{7pt}%
\vspace{2pt}\vspace{2pt}%
}
{%
\vspace{2pt}\end{adjustwidth}\endMakeFramed%
}
% ------------------******-------------------
\newenvironment{Theorem}{%
\def\FrameCommand{%
\hspace{1pt}%
{\color{BlueViolet}\vrule width 2pt}%
{\color{yellowshade}\vrule width 4pt}%
\colorbox{yellowshade}%
}%
\MakeFramed{\advance\hsize-\width\FrameRestore}%
\noindent\hspace{-4.55pt}% disable indenting first paragraph
\begin{adjustwidth}{}{7pt}%
\vspace{2pt}\vspace{2pt}%
}
{%
\vspace{2pt}\end{adjustwidth}\endMakeFramed%
}
% ------------------******-------------------
\newtcolorbox{marker}[1][]{
  enhanced,
  before skip=2mm,after skip=3mm,
  boxrule=0.4pt,left=5mm,right=2mm,top=1mm,bottom=1mm,
  colback=yellow!50,
  colframe=yellow!20!black,
  sharp corners,rounded corners=southeast,arc is angular,arc=3mm,
  underlay={%
    \path[fill=tcbcolback!80!black] ([yshift=3mm]interior.south east)--++(-0.4,-0.1)--++(0.1,-0.2);
    \path[draw=tcbcolframe,shorten <=-0.05mm,shorten >=-0.05mm] ([yshift=3mm]interior.south east)--++(-0.4,-0.1)--++(0.1,-0.2);
    \path[fill=yellow!50!black,draw=none] (interior.south west) rectangle node[white]{\Huge\bfseries !} ([xshift=4mm]interior.north west);
  },
  drop fuzzy shadow,#1
}


  \begin{document}
\else
\fi

\begin{Define}
\begin{define}
  Fix an algebraic structure $\mathbb{R}$ and a set of polynomials $\mathcal{F} = \{F_i\}.$ Let $d$ be a non-negative integer. A $d$-design for $\mathbb{F}$ is a mapping $D$ from $\mathbb{R}$-polynomials of degree $\leq d$ to $\mathbb{R}$ such that:
  \begin{enumerate}
  \item $D(1) = 1$
  \item $D$ is linear, which means $D(X + Y) = D(X) + D(Y)$.
  \item $D(Q\cdot F_i) = 0$ for any polynomial $Q$ and $F_i \in \mathbb{F}$ such that $deg(Q) + deg(F_i) \leq d$.
  \item $D(x_i^2Q) = D(x_iQ)$, where $x$ is a variable in $R[x_1, x_2, ..., x_n]$ and $deg(Q) < d-1$.    
  \end{enumerate}
\end{define}
\end{Define}

Notice that the mapping $D$ is actually determined by the values it assigns to monomials of degree less than $d$, since it is a linear mapping and any polynomial can be decomposed into some monomials. Furthermore, due to property 4, it suffices to consider only multilinear monomials when determining the value of $D$. Alternatively, we can assume that $\mathcal{F}$ includes $x_i(x_i-1)$ for all $i$, in which case property 3 implies property 4.

Naturally, for multilinear monomials, we can view them as conjunctions of the atomic statements represented by the propositional variables in the monomial. Furthermore, we found there is a connection between $d$-designs and Nullstellensatz refutation of degree $d$. Next, two theorems describe this connection.

\begin{Theorem}
  \begin{theorem}
    If $\mathcal{F}$ exists a $d$-design, then there cannot be a Nullstellensatz refutation of degree $\leq d$ to refutate $\mathcal{F}$.
  \end{theorem}
\end{Theorem}

\begin{Proof}
  \begin{proof}
    If there exists a Nullstellensatz refutation of degree $\leq d$, then $$1 = \sum_i P_i\cdot F_i + \sum_j Q_j\cdot (x_j^2 - x_j)$$. Using the $d$-design for $\mathcal{F}$, $$D(1) = 1 \neq 0 = \sum_i D(P_i \cdot F_i) + \sum_j D(Q_jx_j^2) - D(Q_jx_j)$$.
  \end{proof}
\end{Proof}

A converse of the above theorem, to some extent, also holds:

\begin{Theorem}
  \begin{theorem}
    Suppose the algebraic structure $\mathbb{R}$ is a field. If $\mathcal{F}$ does not have a Nullsltellensatz refutation of degree d, then there is a $d$-design for $\mathcal{F}$.
  \end{theorem}
\end{Theorem}

\begin{Proof}
  \begin{proof}
    Suppose there are $n$ variables used in our proposition. Let $\delta$ be the number of monomials of degree $\leq d$. Then we can calculate $\delta = \sum_{i = 0}^d {{n + i - 1}\choose{i}}$. Recall that ${{n + i - 1} \choose {i}} = {{n + i - 1} \choose {n - 1}}$. It means $\delta$ is not infinite. Then for any polynomial $H$ of degree $\leq d$, it can be viewed as a vector $\mathbf{v_H}$ where each element $a_Q$ is the coefficient of a monomial $Q$. Therefore, $\mathbf{v_H}$ has dimension $\delta$.

Let $\mathcal{G}$ be a set of polynomials in the form $Q \cdot F,$
where $Q$ is a monomial and $F \in \mathcal{F},$ and $\deg(Q \cdot F)
\leq d.$ For simplicity, we consider $\mathcal{F}$ to include all
polynomials of the form $x_i(x_i - 1).$ Each $g_i \in \mathcal{G}$ can
be viewed as a vector $\mathbf{v_{g_i}}$, analogous to what was done
for $H.$

Define a vector $\mathbf{v_1}$ with only one nonzero element on the
constant term. A Nullstellensatz refutation of degree at most $d$
exists if and only if there is an $R$-linear combination of the
vectors $\{\mathbf{v_{g_i}}\}$. Let $\mathbf{M}$ be the $\delta \times
|\mathcal{G}|$ matrix with columns $\mathbf{v_{g_i}},$, where $\delta
= \sum_{i = 0}^d {{n + i - 1}\choose{i}}.$ This is equivalent to
finding a solution $\mathbf{w}$ to the linear equation
$\mathbf{M}\mathbf{w} = \mathbf{v_1}$.

It's equivalent to checking whether the rank of coefficient matrix
$\mathbf{M}$ equals the rank of the augmented matrix $[\mathbf{M},
\mathbf{v_1}]$. Since there is only one nonzero element in
$\mathbf{v_1}$, that element cannot be a linear combination of
zeros. It's therefore equivalent to checking whether the last row of
$\mathbf{M}$ is a linear combination of other rows in $\mathbf{M}$.

Thus, there is a Nullstellensatz refutation of degree $\leq d$ if and
only if the last row of $\mathbf{M}$ is linearly independent of the
rest of the row vectors. If the last row is a linear combination, we
can indeed find a possible $d$-design. Denote each row as
$\mathbf{u_Q}$ where $Q$ corresponds to that row. Suppose the
combination is
\begin{equation}
  \label{eq:linear-comb} \sum_{\deg(Q) \leq d}\alpha_Q\mathbf{u_Q} =
0.
\end{equation} Then we can actually use $\alpha_Q$ as $D(Q)$ for all
monomials $Q$, and $D(1) = 1$.

The rest is to check if $D$ is valid. Since it's sufficient that all
properties hold for monomials, we only need to consider the case for
monomials. For any monomial $Q$ and $F \in \mathcal{F}$, $Q\cdot F$
must be in $\mathcal{G}$. Because this is a column in matrix
$\mathbf{M}$, equation~\eqref{eq:linear-comb} holds for the column
$Q\cdot F$. Then $D(Q\cdot F) = D(\sum_{\deg(Q) \leq d}\alpha_Q
u_{Q}^{Q\cdot F}) = 0$. It's not difficult to check that $D$ is
linear.
  \end{proof}
\end{Proof}


\ifx\allfiles\undefined
  \end{document}
\else
\fi