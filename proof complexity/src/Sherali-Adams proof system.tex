%!TEX root=../main.tex
\ifx\allfiles\undefined
  \documentclass{article}
  \usepackage[a4paper,centering,scale=0.8]{geometry}
  \usepackage{newtxtext}
\usepackage{lipsum} % 该宏包是通过 \lipsum 命令生成一段本文,正式使用时不需要引用该宏包
\usepackage[dvipsnames,svgnames]{xcolor}
\usepackage[strict]{changepage} % 提供一个 adjustwidth 环境
\usepackage{framed} % 实现方框效果
\usepackage{tcolorbox}

% ------------------------------------------
\usepackage{amsfonts}
\usepackage{hyperref}
% -----------------------------------------
\usepackage{amsmath}
\usepackage{amsthm}
\newtheorem{define}{Definition}[section]
\newtheorem{theorem}{Theorem}[section]
\newtheorem{lemma}{Lemma}
% -----------------------------------------

\tcbuselibrary{most}
% environment derived from framed.sty: see leftbar environment definition
\definecolor{greenshade}{rgb}{0.90,0.99,0.91} % 文本框颜色
\definecolor{redshade}{rgb}{1.00,0.90,0.90}
\definecolor{yellowshade}{rgb}{0.8,0.8,1.0}
\definecolor{lemmashade}{rgb}{0.9,0.9,1}
% ------------------******-------------------

\newenvironment{Define}[1][]
{%
\def\FrameCommand{%
\hspace{1pt}%
{\color{LightCoral}\vrule width 2pt}%
{\color{redshade}\vrule width 4pt}%
\colorbox{redshade}%
}%
\MakeFramed{\advance\hsize-\width\FrameRestore}%
\noindent\hspace{-4.55pt}% disable indenting first paragraph
\begin{adjustwidth}{}{7pt}%
  \vspace{2pt}\vspace{2pt}%
\begin{define}[#1]
}
{%
\end{define}
n\vspace{2pt}\end{adjustwidth}\endMakeFramed%
}
% ------------------******-------------------
\newenvironment{Proof}
{%
\def\FrameCommand{%
\hspace{1pt}%
{\color{Green}\vrule width 2pt}%
{\color{greenshade}\vrule width 4pt}%
\colorbox{greenshade}%
}%
\MakeFramed{\advance\hsize-\width\FrameRestore}%
\noindent\hspace{-4.55pt}% disable indenting first paragraph
\begin{adjustwidth}{}{7pt}%
  \vspace{2pt}\vspace{2pt}%
\begin{proof}
}
{%
\end{proof}
\vspace{2pt}\end{adjustwidth}\endMakeFramed%
}
% ------------------******-------------------
\newenvironment{Theorem}[1][]
{%
\def\FrameCommand{%
\hspace{1pt}%
{\color{BlueViolet}\vrule width 2pt}%
{\color{yellowshade}\vrule width 4pt}%
\colorbox{yellowshade}%
}%
\MakeFramed{\advance\hsize-\width\FrameRestore}%
\noindent\hspace{-4.55pt}% disable indenting first paragraph
\begin{adjustwidth}{}{7pt}%
  \vspace{2pt}\vspace{2pt}%
\begin{theorem}[#1]
}
{%
\end{theorem}
\vspace{2pt}\end{adjustwidth}\endMakeFramed%
}

% -------------------------------------------
\newenvironment{Lemma}[1][]
{%
\def\FrameCommand{%
\hspace{1pt}%
{\color{BlueViolet}\vrule width 2pt}%
{\color{yellowshade}\vrule width 4pt}%
\colorbox{lemmashade}%
}%
\MakeFramed{\advance\hsize-\width\FrameRestore}%
\noindent\hspace{-4.55pt}% disable indenting first paragraph
\begin{adjustwidth}{}{7pt}%
  \vspace{2pt}\vspace{2pt}%
\begin{lemma}[#1]
}
{%
\end{lemma}
\vspace{2pt}\end{adjustwidth}\endMakeFramed%
}
% ------------------******-------------------
\newtcolorbox{marker}[1][]{
  enhanced,
  before skip=2mm,after skip=3mm,
  boxrule=0.4pt,left=5mm,right=2mm,top=1mm,bottom=1mm,
  colback=yellow!50,
  colframe=yellow!20!black,
  sharp corners,rounded corners=southeast,arc is angular,arc=3mm,
  underlay={%
    \path[fill=tcbcolback!80!black] ([yshift=3mm]interior.south east)--++(-0.4,-0.1)--++(0.1,-0.2);
    \path[draw=tcbcolframe,shorten <=-0.05mm,shorten >=-0.05mm] ([yshift=3mm]interior.south east)--++(-0.4,-0.1)--++(0.1,-0.2);
    \path[fill=yellow!50!black,draw=none] (interior.south west) rectangle node[white]{\Huge\bfseries !} ([xshift=4mm]interior.north west);
  },
  drop fuzzy shadow,#1
}

% -------------------------------------------
\newcommand{\authnote}[2]{[{\color{red}\textbf{#1:}}~{\color{blue} #2}]}
\newcommand{\Nie}[1]{\authnote{Xinhao Nie}{#1}}
\newcommand{\TBD}[1]{\authnote{TBD}{#1}}
\newcommand{\TL}[1]{\authnote{ToLearn}{#1}}

\newcommand{\RR}{\mathbb{R}}
\newcommand{\FF}{\mathbb{F}}
\newcommand{\QQ}{\mathbb{Q}}

\newcommand{\xx}{\mathbf{x}}
\newcommand{\yy}{\mathbf{y}}
\newcommand{\vv}{\mathbf{v}}
\newcommand{\uu}{\mathbf{u}}
\newcommand{\ww}{\mathbf{w}}
\newcommand{\ee}{\mathbf{e}}

\newcommand{\opA}{\mathcal{A}}

\newcommand{\spanof}[1]{\operatorname{span}\left\{#1\right\}}
\DeclareMathOperator{\rank}{rank}
\DeclareMathOperator{\charac}{char}
\DeclareMathOperator{\tr}{Tr}
\DeclareMathOperator{\Spec}{Spec}
  \begin{document}
\else
\fi

Originally, it was designed to handle mixed linear programming. Amazing!
\subsection{Mixed-integer Zero-one Programming}
\subsubsection{Relaxation and Linearization}
Consider a linear mixed-integer zero-one programming problem whose feasible region is given as follows:
\begin{equation}
  \begin{aligned}
    \label{origin-problem}
    X = \{(x,y): &\sum_{j=1}^n \alpha_{rj}x_j + \sum_{k=1}^m \gamma_{rk}y_k \geq \beta_r ,\forall r \in [R],\\ &x_j \in \{0,1\} ,\forall j \in [n], \\ &0\leq y_k \leq 1, \forall k \in [m] \}
  \end{aligned}
\end{equation}

One classical approach involves relaxing the integrality constraint by simply treating the variables as continuous real values rather than integers, thereby allowing for an optimal but potentially infeasible solution. The Sherali-Adams reformulation-linearization is another approach.

\begin{Define}
  For any $d \in [N]$, let us define the polynomial factors of degree $d$ as $$F_d(J_1, J_2) = (\prod_{j \in J_1} x_j)(\prod_{j \in J_2} (1-x_j))$$ for each $J_1, J_2 \subseteq [N]$ such that $J_1 \cap J_2 = \phi,$ and $|J_1 \cup J_2| = d.$ Sepcially, $F_0(\phi, \phi) = 1.$
\end{Define}

Sherali and Adams use each polynomial factor of differing degree to construct a hierarchy of relaxations. In the description of relaxation algorithm, we just fix a $d$. We multiply $F_d(J_1, J_2)$ on both sides and remove the integerality constraints, but, to some extent, hidenly add constraints $x_i(x_i - 1) = 0$. Then the programming problem \eqref{origin-problem} would be updated to
\begin{equation}
  \begin{aligned}
    \label{con1}
    (\sum_{j \in J_1} \alpha_{rj})F_d(J_1, J_2) - \beta_r F_d(J_1, J_2) + \sum_{j \in [N]\setminus (J_1 \cup J_2)}\alpha_{rj}F_{d+1}(J_1 + j, J_2) + \sum_{k=1}^m \gamma_{rk}y_kF_d(J_1, J_2) \geq 0, \forall r\in [R], \forall F_d.
  \end{aligned}
\end{equation}


Because we remove integerality constraints, it's suffice to restrict $0 \leq x,y \leq 1$. Thus, we have another two constraints
\begin{equation}
  \begin{gathered}
    \label{con2}
  F_D(J_1, J_2) \geq 0, D = min(d+1, N), \forall F_D \\
    F_d(J_1, J_2) \geq y_kF_d(J_1, J_2) \geq 0, \forall F_d, \forall k\in [m].
  \end{gathered}
\end{equation}

\begin{Define}
  Viewing the constraints above in expanded form as a sum of monomials, linearize them by substituting the following variables for the corresponding nonlinear terms $\forall J \subseteq [N],\forall k \in [m], w_J = \prod_{j \in J}x_j, v_{Jk} = y_k\prod_{j \in J}x_j.$ Specially, $w_{\phi} = 1.$ Furthermore, denoting by $f_d(J_1, J_2)$ and $f_d^k(J_1, J_2)$ the respective linearized forms of the polynomial expressions $F_d(J_1, J_2)$ and $y_kF_d(J_1, J_2)$.
\end{Define}

The original problem of finding the optimal $(X, Y)$ is transformed into a new problem of finding the optimal $(W, V)$. Actually, $W$ includes $w_j = x_j$ and $V$ includes $v_{\phi k} = y_k.$ For simplicity, we can denote the solution as $(X,Y,W,V).$ We then need to
demonstrate the validity of this relaxation.

\subsubsection{Validity}

\begin{Theorem}
  \begin{theorem}
    $Conv(X) \subseteq X_{P_n} \subseteq X_{P_{n-1}} \subseteq ... \subseteq X_{P_1} \subseteq X_{P_0} = X_{0}$ where $X_{P_d}$ is the corresponding $(\mathbf{x}, \mathbf{y})$ to $X_d = \{(\mathbf{x}, \mathbf{y}, \mathbf{w}, \mathbf{v})\}.$ In particular, $X_{P_d} \cap \{(x,y): x \ is \ binary\} = X$ for all $d \in [N].$
  \end{theorem}
\end{Theorem}

\begin{Proof}
  \begin{proof}
    First, we show $X_{P_{d}} \subseteq X_{P_{d-1}}.$
    $$F_d(J_1 + j, J_2) \geq 0, F_d(J_1, J_2 + j) \geq 0 \Rightarrow F_{d-1}(J_1, J_2) \geq 0$$
    Then constraints~\eqref{con2} hold. For constraints~\eqref{con1}, all terms are analogous except $\sum_{j\in N\setminus{J_1\cup J_2}} \alpha_{r_j} F_{d+1}(J_1 + j, J_2).$ We choose pairs $(J_1 + t, J_2)$ and $(J_1, J_2 + t)$ where $t \not \in J_1 \cup J_2.$ Then $\sum_{j \in N\setminus{J_1\cup J_2\cup t}}(F_{d+2}(J_1 + t + j, J_2)) + \sum_{j \in N\setminus{J_1\cup J_2\cup t}}(F_{d+2}(J_1 + j, J_2 + t)) + F_{d + 1}(J_1 + t, J_2) = \sum_{j\in N\setminus{J_1\cup J_2}} \alpha_{r_j} F_{d+1}(J_1 + j, J_2).$ Then we can conclude $X_{P_{d}} \subseteq X_{P_{d-1}}.$

    Second, let us show that $Conv(X) \subseteq X_{P_n}.$ $X \subseteq X_{P_n}$ is trivial. And $X_{P_n}$ is convex. Thus $Conv(X) \subseteq X_{P_n}$. In particular, $X = Conv(X) \cap \{ (x,y): x \ is \ binary\} = X_0 \cap \{ (x,y): x \ is \ binary\}$ because $X_0$ just removes integrality restrictions. Finally, it's suffice that $X_{P_d} \cap \{(x,y): x \ is \ binary\} = X$ for all $d \in [N].$
  \end{proof}
\end{Proof}

\subsubsection{Convexity}
\begin{Theorem}
  \begin{theorem}
    $X_{P_n} = Conv(X)$
  \end{theorem}
\end{Theorem}


\subsection{Application on Proof System}



\ifx\allfiles\undefined
  \end{document}
\else
\fi