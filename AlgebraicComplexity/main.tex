\documentclass[12pt]{article}

\usepackage[a4paper,centering,scale=0.8]{geometry}
\usepackage[affil-it]{authblk}

%------------------ImportEnv---------------
\usepackage{newtxtext}
\usepackage{lipsum} % 该宏包是通过 \lipsum 命令生成一段本文,正式使用时不需要引用该宏包
\usepackage[dvipsnames,svgnames]{xcolor}
\usepackage[strict]{changepage} % 提供一个 adjustwidth 环境
\usepackage{framed} % 实现方框效果
\usepackage{tcolorbox}

% ------------------------------------------
\usepackage{amsfonts}
\usepackage{hyperref}
% -----------------------------------------
\usepackage{amsmath}
\usepackage{amsthm}
\newtheorem{define}{Definition}[section]
\newtheorem{theorem}{Theorem}[section]
% -----------------------------------------

\tcbuselibrary{most}
% environment derived from framed.sty: see leftbar environment definition
\definecolor{greenshade}{rgb}{0.90,0.99,0.91} % 文本框颜色
\definecolor{redshade}{rgb}{1.00,0.90,0.90}
\definecolor{yellowshade}{rgb}{0.90,0.90,1.0}
% ------------------******-------------------
% 注意行末需要把空格注释掉,不然画出来的方框会有空白竖线
\newenvironment{Define}{%
\def\FrameCommand{%
\hspace{1pt}%
{\color{LightCoral}\vrule width 2pt}%
{\color{redshade}\vrule width 4pt}%
\colorbox{redshade}%
}%
\MakeFramed{\advance\hsize-\width\FrameRestore}%
\noindent\hspace{-4.55pt}% disable indenting first paragraph
\begin{adjustwidth}{}{7pt}%
\vspace{2pt}\vspace{2pt}%
}
{%
\vspace{2pt}\end{adjustwidth}\endMakeFramed%
}
% ------------------******-------------------
\newenvironment{Proof}{%
\def\FrameCommand{%
\hspace{1pt}%
{\color{Green}\vrule width 2pt}%
{\color{greenshade}\vrule width 4pt}%
\colorbox{greenshade}%
}%
\MakeFramed{\advance\hsize-\width\FrameRestore}%
\noindent\hspace{-4.55pt}% disable indenting first paragraph
\begin{adjustwidth}{}{7pt}%
\vspace{2pt}\vspace{2pt}%
}
{%
\vspace{2pt}\end{adjustwidth}\endMakeFramed%
}
% ------------------******-------------------
\newenvironment{Theorem}{%
\def\FrameCommand{%
\hspace{1pt}%
{\color{BlueViolet}\vrule width 2pt}%
{\color{yellowshade}\vrule width 4pt}%
\colorbox{yellowshade}%
}%
\MakeFramed{\advance\hsize-\width\FrameRestore}%
\noindent\hspace{-4.55pt}% disable indenting first paragraph
\begin{adjustwidth}{}{7pt}%
\vspace{2pt}\vspace{2pt}%
}
{%
\vspace{2pt}\end{adjustwidth}\endMakeFramed%
}
% ------------------******-------------------
\newtcolorbox{marker}[1][]{
  enhanced,
  before skip=2mm,after skip=3mm,
  boxrule=0.4pt,left=5mm,right=2mm,top=1mm,bottom=1mm,
  colback=yellow!50,
  colframe=yellow!20!black,
  sharp corners,rounded corners=southeast,arc is angular,arc=3mm,
  underlay={%
    \path[fill=tcbcolback!80!black] ([yshift=3mm]interior.south east)--++(-0.4,-0.1)--++(0.1,-0.2);
    \path[draw=tcbcolframe,shorten <=-0.05mm,shorten >=-0.05mm] ([yshift=3mm]interior.south east)--++(-0.4,-0.1)--++(0.1,-0.2);
    \path[fill=yellow!50!black,draw=none] (interior.south west) rectangle node[white]{\Huge\bfseries !} ([xshift=4mm]interior.north west);
  },
  drop fuzzy shadow,#1
}

% -------------------------------------------
\newcommand{\authnote}[2]{[{\color{red}\textbf{#1:}}~{\color{blue} #2}]}
\newcommand{\Nie}[1]{\authnote{Xinhao Nie}{#1}}
\newcommand{\TBD}[1]{\authnote{TBD}{#1}}
\newcommand{\TL}[1]{\authnote{ToLearn}{#1}}
% -----------------ComplexityClass---------
% --------ComplexityClass----------
\def\fp/{\textup{\textsf{FP}}}
\def\fnp/{\textup{\textsf{FNP}}}
\def\tfnp/{\textup{\textsf{TFNP}}}
\def\np/{\textup{\textsf{NP}}}
\def\npc/{\textup{\testsf{NP}-complete}}
\def\conp/{\textup{\textsf{coNP}}}
\def\p/{\textup{\textsf{P}}}

% ----------Problem----------------
\def\TAUT/{\textup{\textsf{TAUT}}}
\def\allfiles{}
\begin{document}
\title{Notes about Algebraic Complexity}
\author{Xinhao Nie}
\affil{Aarhus University}
\maketitle

\section{\SqrtSum}

\begin{Define}
  Given positive integers $a_1, \dots a_n$ and an integer $t,$ we want to decide whether $\sum_{i=1}^n\sqrt{a_i} \leq t.$
\end{Define}


\begin{Lemma}
  We can test whether $\sum_{i=1}^n \sqrt{a_i} = t$ in polynomial
  time.
\end{Lemma}

\begin{Proof}
  The paper 'Decreasing the Nesting Depth of Expressions
  Involving Square Roots' implicitly implies this result in its
  section~5.

  First denest any individual radical that denests, which means for
  all $a_i$ is not perfect sqaure number. Then we choose pair of
  remaining radicals $\sqrt{a_i}$ and $\sqrt{a_j}$ if their product
  denests in $\QQ$. If $\sqrt{a_i} \sqrt{a_j}$ denests as
  $k \in \QQ$, then replace $\sqrt{a_i} + \sqrt{a_j}$ by
  $$
  (1 + \frac{k}{a_i})\sqrt{a_i} \in \QQ(\sqrt{a_i}),
  $$
  and iterate the process of looking for a pair of radicals that
  desnests. Its correctness can be easily checked.

  If at some point no product of a pair of radicals denests, then we
  claim that the entire linear combination could not denest. Next
  we should prove this claim. The rough idea is that we show
  $\forall \sqrt{a_i} = \lambda \ee_j$ where $\lambda \in \QQ$ and
  $\langle \ee_1, \dots,
  \ee_{2^m} \rangle$ span the vector space $\QQ(\sqrt{a_1}, \dots,
  \sqrt{a_m})$ over $\QQ$ if $
  [\QQ(\sqrt{a_1}, \dots, \sqrt{a_n}):\QQ] = 2^m$. Concretely, all
  basis $\ee$ are the subproducts of $m$ square roots. Then, we can
  concluded that if $\sum_{i=1}^n \sqrt{a_i} = t$, each square root
  must be eliminated because they are distinct basis. In other
  words, if there is no product of a pair of radicals denests, then
  it cannot equal to a rational number.

  Together with some fact about basis of extension $\QQ$ over $\QQ$,
  above statement is directly showed by Lemma~1 in the original
  paper. But it's more general, here we beriefly prove
  what we use.
  Suppose that $[\QQ(\sqrt{a_1},\dots,\sqrt{a_m}):\QQ] = 2^m$ and
  prove by induction if
  $\sqrt{a_{m+1}} \in \QQ(\sqrt{a_1},\dots,\sqrt{a_m})$ then
  $\sqrt{a_{m+1}} = \lambda \sqrt{a_1...a_m}$.

  Basis(0): $\lambda = \sqrt{a_1}$.
  Hypothesis(m-1) and Induction(m): Write $\sqrt{a_{m+1}} =
  b\sqrt{a_{m}} + c$, where $b, c \in \QQ(\sqrt{a_1},\dots,
  \sqrt{a_{m-1}})$. By squaring,
  $$
  a_{m+1} = b^2a_m + c^2 + 2bc\sqrt{a_m},
  $$
  which indicates $c = 0$, since $a_{m+1} \in \QQ$. Then, we have
  $$
  \sqrt{a_{m+1}a_m} = ba_{m} \in \QQ(\sqrt{a_1},\dots,\sqrt{a_{m-1}}).
  $$
  By hypothesis, $\sqrt{a_{m+1}a_m} = \lambda \sqrt{a_1...a_{m-1}}$,
  which implies $\sqrt{a_{m+1}} = \lambda a_m^{-1}
  \sqrt{a_1...a_{m-1}a_{m}}$. Thus, we can write each $\sqrt{a_i}
  = \lambda \sqrt{a_1...a_m}$. Moreover, the $2^m$ subproducts of
  $\sqrt{a_1}, \dots, \sqrt{a_m}$ are the basis of $\QQ(\sqrt{a_1},
  \dots,\sqrt{a_n})$ over $\QQ$. \href{https://math.stackexchange.com/questions/30687/the-square-roots-of-different-primes-are-linearly-independent-over-the-field-of}{The proof is here.}

  
\end{Proof}

\end{document}