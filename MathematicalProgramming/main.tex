\documentclass[12pt]{article}

\usepackage[a4paper,centering,scale=0.8]{geometry}
\usepackage[affil-it]{authblk}

%------------------ImportEnv---------------
\usepackage{newtxtext}
\usepackage{lipsum} % 该宏包是通过 \lipsum 命令生成一段本文,正式使用时不需要引用该宏包
\usepackage[dvipsnames,svgnames]{xcolor}
\usepackage[strict]{changepage} % 提供一个 adjustwidth 环境
\usepackage{framed} % 实现方框效果
\usepackage{tcolorbox}

% ------------------------------------------
\usepackage{amsfonts}
\usepackage{hyperref}
% -----------------------------------------
\usepackage{amsmath}
\usepackage{amsthm}
\newtheorem{define}{Definition}[section]
\newtheorem{theorem}{Theorem}[section]
% -----------------------------------------

\tcbuselibrary{most}
% environment derived from framed.sty: see leftbar environment definition
\definecolor{greenshade}{rgb}{0.90,0.99,0.91} % 文本框颜色
\definecolor{redshade}{rgb}{1.00,0.90,0.90}
\definecolor{yellowshade}{rgb}{0.90,0.90,1.0}
% ------------------******-------------------
% 注意行末需要把空格注释掉,不然画出来的方框会有空白竖线
\newenvironment{Define}{%
\def\FrameCommand{%
\hspace{1pt}%
{\color{LightCoral}\vrule width 2pt}%
{\color{redshade}\vrule width 4pt}%
\colorbox{redshade}%
}%
\MakeFramed{\advance\hsize-\width\FrameRestore}%
\noindent\hspace{-4.55pt}% disable indenting first paragraph
\begin{adjustwidth}{}{7pt}%
\vspace{2pt}\vspace{2pt}%
}
{%
\vspace{2pt}\end{adjustwidth}\endMakeFramed%
}
% ------------------******-------------------
\newenvironment{Proof}{%
\def\FrameCommand{%
\hspace{1pt}%
{\color{Green}\vrule width 2pt}%
{\color{greenshade}\vrule width 4pt}%
\colorbox{greenshade}%
}%
\MakeFramed{\advance\hsize-\width\FrameRestore}%
\noindent\hspace{-4.55pt}% disable indenting first paragraph
\begin{adjustwidth}{}{7pt}%
\vspace{2pt}\vspace{2pt}%
}
{%
\vspace{2pt}\end{adjustwidth}\endMakeFramed%
}
% ------------------******-------------------
\newenvironment{Theorem}{%
\def\FrameCommand{%
\hspace{1pt}%
{\color{BlueViolet}\vrule width 2pt}%
{\color{yellowshade}\vrule width 4pt}%
\colorbox{yellowshade}%
}%
\MakeFramed{\advance\hsize-\width\FrameRestore}%
\noindent\hspace{-4.55pt}% disable indenting first paragraph
\begin{adjustwidth}{}{7pt}%
\vspace{2pt}\vspace{2pt}%
}
{%
\vspace{2pt}\end{adjustwidth}\endMakeFramed%
}
% ------------------******-------------------
\newtcolorbox{marker}[1][]{
  enhanced,
  before skip=2mm,after skip=3mm,
  boxrule=0.4pt,left=5mm,right=2mm,top=1mm,bottom=1mm,
  colback=yellow!50,
  colframe=yellow!20!black,
  sharp corners,rounded corners=southeast,arc is angular,arc=3mm,
  underlay={%
    \path[fill=tcbcolback!80!black] ([yshift=3mm]interior.south east)--++(-0.4,-0.1)--++(0.1,-0.2);
    \path[draw=tcbcolframe,shorten <=-0.05mm,shorten >=-0.05mm] ([yshift=3mm]interior.south east)--++(-0.4,-0.1)--++(0.1,-0.2);
    \path[fill=yellow!50!black,draw=none] (interior.south west) rectangle node[white]{\Huge\bfseries !} ([xshift=4mm]interior.north west);
  },
  drop fuzzy shadow,#1
}

% -------------------------------------------
\newcommand{\authnote}[2]{[{\color{red}\textbf{#1:}}~{\color{blue} #2}]}
\newcommand{\Nie}[1]{\authnote{Xinhao Nie}{#1}}
\newcommand{\TBD}[1]{\authnote{TBD}{#1}}
\newcommand{\TL}[1]{\authnote{ToLearn}{#1}}
% -----------------ComplexityClass---------
% --------ComplexityClass----------
\def\fp/{\textup{\textsf{FP}}}
\def\fnp/{\textup{\textsf{FNP}}}
\def\tfnp/{\textup{\textsf{TFNP}}}
\def\np/{\textup{\textsf{NP}}}
\def\npc/{\textup{\testsf{NP}-complete}}
\def\conp/{\textup{\textsf{coNP}}}
\def\p/{\textup{\textsf{P}}}

% ----------Problem----------------
\def\TAUT/{\textup{\textsf{TAUT}}}
\def\allfiles{}
\begin{document}
\title{Notes about Mathematical Programming}
\author{Xinhao Nie}
\affil{Aarhus University}
\maketitle

\section{Linear Programming}

There are two different definitions about
standard form of linear programming. One is
\begin{align*}
  \text{maximize} \quad & c^\top \xx \\
  \text{subject to} \quad & A\xx \leq b \\
                        & \xx \geq 0,
\end{align*}
while another one is:
\begin{Define}[Primal Problem]
    \normalfont
\begin{align*}
  \text{minimize} \quad & c^\top \xx \\
  \text{subject to} \quad & A\xx = b \\
                        & \xx \geq 0.
\end{align*}
\end{Define}
Obviously, they are euqal. Actually, it doesn't
really matter whether an LP is formulated as a
minimization or maximization problem. However,
whether the constraints are written with $\geq$
or $=$ does affect the way the problem is
understood.

When we are trying to use simplex method, the
first one is more suitable because we can
easily import slack varibles and do pivot
rule. Things go complicated when we try to
prove strong duality theorem as what we will
show next.

\subsection{Duality}

According to minimization or maximization
in prime problem, we can give corresponding
dual problem as follows:

\begin{Define}[Dual Problem]
    \normalfont
\begin{align*}
  \text{maximize} \quad & b^\top \yy \\
  \text{subject to} \quad & A^\top \yy \leq c.
\end{align*}
\end{Define}

\subsubsection{Weak Duality Theorem}

Many people view it as one of the most
remarkable and useful features of linear
programming because of the relation between
optimal value in prime problem and that in
dual problem.

\begin{Theorem}[Weak Duality]
    The optimal value in corresponding dual
    problem is
    always less than the optimal value in
    prime problem.
\end{Theorem}

\begin{Proof}
  \begin{align*}
      \forall \yy,\xx. b^\top \yy =
      (A\xx)^\top \yy
      =\xx^\top A^\top \yy \leq \xx^\top c =
      c^\top \xx.
  \end{align*}
\end{Proof}

\subsubsection{Strong Duality Theorem}

The proof of Strong Duality Theorem depends on the standard form LP, thus, it vaies from different definitions. The following proof is based on our definition. To this end, we need soem useful lemmas in discrete optimization.

\begin{Lemma}[Farkas' Lemma]
  \label{FarkasLemma}
  Let $A \in \mathbb{R}^{m \times n}$ be a matrix and $b \in \mathbb{R}^m$ be a vector. The system $Ax = b, x \geq 0$ has a solution if and only if for all $\lambda \in \mathbb{R}^m$ with $\lambda^\top A \geq 0$ one has $\lambda^\top b \geq 0$.

  \normalfont  \Nie{There are a number of slightly different formulations of the lemma in the literature. Notice that the original one is exactly one of the above two assertion is true. Intuitively \eqref{FarkasLemma} is weaker, however, $\lambda = 0$ is always a trivial solution of $\lambda^\top A \geq 0$. Thus, they are equivalent. }
\end{Lemma}

\begin{Proof}
  \TL{It's a special variant of Separation Theorem. As a result in geometry, I believe it's straightforward but hard to prove. TODO.}
\end{Proof}

An interesting and easy variant of Farkas' lemma is either $Ax = b$ has a solution $x \in \RR^n$, or $A^\top y = 0$ has a solution $y \in \RR^m$ with $b^\top y \neq 0$. The proof is quite simple. If $Ax = b$ has a solution, $\rank{A} = \rank{[A|b]}$. Thus $b$ is a linear combination of $A$. Then there exists some non-trivial numbers $\lambda_1, \lambda_2,..., \lambda_n$ s.t. $\forall i \in [m]. b_i = \lambda_1 A_{i,1} + \lambda_2 A_{i,2} + \dots + \lambda_n A_{i,n}$. If there exists $y$ s.t. $\forall j\in [n], y_1A_{1,j} + y_2A_{2,j} + \dots + y_mA_{m,j} = 0$, we have $\sum_{i \in [m]} y_ib_i = \sum_{i \in [m]} \sum_{j \in [n]} \lambda_jy_iA_{i,j} = \sum_{j \in [n]} \lambda_j \sum_{i \in [m]} y_iA_{i,j} = 0$. Conversely, suppose there is a solution $x^*$. $0 = y^\top A = y^\top Ax^* = y^\top b \neq 0$ leads to a contradiction. It's named as \emph{Fredholm alternative}.


\begin{Lemma}[Second Variant of Farkas' Lemma]
  Let $A \in \RR^{m\times n}$ and $b \in \RR^m$. The system $Ax \leq b$ has a solution if and only if for all $\lambda \in \RR^m \geq 0$ with $\lambda^\top A = 0$ one has $\lambda^\top b \geq 0$.
\end{Lemma}
\begin{Proof}
  $\Rightarrow$: If there is a feasible $x^*$ s.t. $Ax^* \leq b$, we have $\forall \lambda \geq 0. \lambda^\top Ax \leq \lambda^\top b$. It implies $\lambda^\top b \geq 0$.
  
  $\Leftarrow$: Firstly, $Ax^+ - Ax^- + z = b, x^+,x^-,z \geq 0$ has a solution if and only if $Ax \leq b$ has a solution. We change it to matrix form as $[A|-A|I_m]X=b, X \geq 0$. Using Farkas' Lemma \eqref{FarkasLemma}, we have it has a solution if and only if $\forall \lambda \in \RR^m$ with $\lambda[A|-A|I_m] \geq 0$ one has $\lambda^\top b \geq 0$. It's equivalent to $\lambda A = 0$ and $\lambda \geq 0$.
\end{Proof}

\begin{Theorem}
  If both prime problem and dual problem are feasible, then they have same optimal value.
\end{Theorem}

Suppose there are $\xx^*$ and $\yy^*$ which are optimal varibales in each problem. $A\xx^* = b$ and $\xx^* \geq 0$ hold. Let's consider the solution space of $A \xx^* = b.$ If $A$ is full-rank, $\xx^* = A^{-1}b$ is a unique solution because $\xx^*$ is a solution. In this case, the optimal value of prime problem is $c^\top A^{-1} b$. Let $\yy' = (A^{-1})^\top c.$ We can verify that $b^\top\yy' = c^\top A^{-1} b$ and $A^\top y = A^\top (A^{-1})^\top c \leq c.$ Together with Weak Duality Theorem, there is no gap between two optimal values.

  Now, we consider the case that $A$ is not full-rank. Actually, it could be proved using simplex method. The result of choosing basic variables and non-basic variables is consist of a full-rank submatrix. Using a similar analysis as above, we can conclude it. But it depends on the relation between dual variables and prime variables which also relates to the definition.  Here we can give a more mathematical proof using Farkas' Lemma.

\begin{Proof}
  Suppose the maximal value of dual problem is
  $\delta$. There is no feasible solution in
  $b^\top y \geq \delta + \epsilon, A^\top y
  \leq c$. Change it into
  $\begin{bmatrix} -b^\top \\ A^\top
   \end{bmatrix} y \leq
   \begin{bmatrix} - \delta - \epsilon \\
     c \end{bmatrix}$.
   Using the second variant of Farkas' lemma
   in the new problem, we know that there
   exists $\lambda \in \RR^{m+1} \geq 0$ with
   $\lambda^\top
   \begin{bmatrix} -b^\top \\
     A^\top
   \end{bmatrix} = 0$ one has
   $\lambda^\top
   \begin{bmatrix} - \delta - \epsilon \\
     c \end{bmatrix} < 0$.
   Be carefully here! Generally the negation
   of $\geq$ is not $<$ because it's
   compenant-wise operation. But here it's
   just one number.

   Suppose $\lambda =
   \begin{bmatrix} \lambda_1 \\ \lambda'
   \end{bmatrix}$. We can show that
   $\lambda_1 \geq 0$. Using the second
   variant of Farkas' lemma in original dual
   problem, we have for all $\lambda' \geq 0$
   with $\lambda'^\top A^\top = 0$ one has
   $\lambda^\top c \geq 0$. If
   $\lambda_1 = 0$, it conflicts with previous
   result. Furthermode, by scaling, we can
   assume $\lambda_1 = 1$. One has
   $\lambda'^\top A^\top = b^\top$ and
   $\lambda' c < \delta + \epsilon$. Here
   $\lambda'$ is a valid variable for primal
   problem with objective value less than
   $\delta + \epsilon$. Together with weak
   duality theorem, we get a infimum.
  
\end{Proof}

\subsubsection{Complementary Slackness}

\begin{Theorem}
  As a conclusion of strong and weak duality
  theorem, we have
  $\forall i \in [n]. x_i^*(c - A^\top \yy^*)_i
  = 0$.
\end{Theorem}

\begin{Proof}
  $b^\top \yy^* = c^\top \xx^*
  \geq \yy^{*\top} A \xx^*
  = \yy^{*\top} b = b^\top \yy^*$. Thus,
  $\forall i \in [n]. x_i^*(c - A^\top \yy^*)_i
  = 0$ must hold.
\end{Proof}

\section{Semidefinite Programming}

\subsection{Preliminary}

\subsubsection{Bilinear Form}

In mathematics, a bilinear form is a bilinear
map $V \times V \rightarrow K$ on a vector
space $V$ over a field $K$. In this chapter,
we restrict it's on real number field.

\begin{Define}
  A bilinear form $f: V \times V \rightarrow
  \RR$ is a map such that:
  \begin{itemize}
  \item $f(\alpha \uu + \beta \vv, \ww) =
    \alpha f(\uu, \ww) + \beta f(\vv, \ww)$
  \item $f(\ww, \alpha \uu + \beta \vv) =
    \alpha f(\ww, \uu) + \beta f(\ww, \vv)$,
  \end{itemize}
  where $\vv, \uu, \ww \in V$ and
  $\alpha, \beta \in \RR$.
\end{Define}

Once we selected the basis of vector space
$V = \langle \ee_1, \ee_2, \dots, \ee_n
\rangle$, the bilinear form $f$ can be
represented by a $n \times n$ matrix since the
bilinear property.
$$
f(\xx, \yy) =
f(\sum_i x_i \ee_i, \sum_j y_j \ee_j) =
\sum_i x_if(\ee_i, \sum_j y_j \ee_j) =
\sum_i \sum_j x_iy_j f(\ee_i, \ee_j) =
\sum_{i,j} x_iy_jf_{ij}.
$$
Thus, we can rewrite $f(\xx,\yy)$ as $\xx^\top
F \yy$. Moreover, given the change-of-basis
matrix, we can calculate corresponding matrix
$F$. Suppose the change-of-basis matrix is
$A = (a_{ij}):
\forall j \in [n].
\ee'_j = \sum_{i=1}^n a_{ij}\ee_i$.
We have $\xx = A \xx'$.\Nie{Be careful,
  it's a little bit confusing.} Then,
\begin{align*}
  \xx'^\top F' \yy' = f'(\xx', \yy') =
  f(\xx,\yy) = \xx^\top F \yy =
  (A \xx')^\top F (A \yy') =
  \xx'^\top A^\top F A \yy'.
\end{align*}
Finally, we get the relation $F' = A^\top F A$.
We call it as \emph{congruence}. One can
observe that $\rank{F} = \rank{F'}$ by the
rank inequility, namely $\rank{AB} \leq
\min\{\rank{A}, \rank{B}\}$, and $A$ is
full-rank. As analyzed above, the rank is the
first invariant we have observed that is
independent of the choice of basis. Next, we
present another invariant that is also
independent of the choice of basis:
symmetry/skew-symmetry.

Since $f(\xx, \yy) = \epsilon f(\yy, \xx)$ and
$\epsilon = \pm 1$, we have
$$
\xx^\top F \yy = f(\xx, \yy) =
\epsilon f(\yy,\xx) = \epsilon f^\top(\yy,\xx)
= \epsilon \xx^\top F^\top \yy.
$$
In other words, $F = \epsilon F^\top$.
Following from congruence, $F' = A^\top F A
= \epsilon A^\top F^\top A = \epsilon F'^\top$.

If $\charac{\RR} \neq 2$, the space of
bilinear forms is the direct sum of symmetric
subspace and skew-symmetric subspace. The
proof follows directly from a simple fact
$f(\xx, \yy) =
\frac{1}{2}(f(\xx, \yy) + f(\yy, \xx)) +
\frac{1}{2}(f(\xx, \yy) - f(\yy, \xx))$.

\subsubsection{Quadratic Form}

\begin{Define}
  $q: V \rightarrow \RR$ is a quadratic form,
  if it meets two conditions:
  \begin{enumerate}
  \item $q(\vv) = q(- \vv),
    \forall \vv \in V$
  \item $f(\xx, \yy) =
    \frac{1}{2}(q(\xx + \yy) - q(\xx) -
    q(\yy))$ is a symmetric bilinear form.
  \end{enumerate}
  Additionally, define $\rank{q} = \rank{f}$.
\end{Define}

One can get the quadratic form from the
corresponding bilinear form and vice versa.
Notice that
$$
f(\xx, -\xx) =
\frac{1}{2}(q(\xx - \xx) - q(\xx) - q(-\xx))
= \frac{1}{2}q(0) - q(\xx),
$$

$$
f(\xx, \xx) = -f(\xx, -\xx)
= q(\xx) - \frac{1}{2}q(0).
$$
Meanwhile, $f(0,0) = \frac{1}{2}q(0) = 0$.
We have $q(\xx) = f(\xx, \xx) =
\xx^\top F \xx = \sum_{i,j}f_{ij}x_ix_j$.

\begin{Define}[Canonical Form]
  For a quadratic form $q$, select some basis
  $\langle \ee_1, \ee_2, \dots, \ee_n \rangle$
  such that $\forall \xx =
  \sum_{i=1}^n x_i\ee_i \in V$ one has
  $q(\xx) = \sum_{i=1}^n f_{ii}x_i^2$. Then,
  it's the \emph{canonical form} of $q$ under
  the canonical basis $\langle \ee_1, \ee_2,
  \dots, \ee_n \rangle$.
\end{Define}

\begin{Theorem}
  For all symmetric bilinear form $f$ or just
  a symmetric matrix $F$, there is a canonical
  basis.
\end{Theorem}

\begin{Proof}
  Prove it by induction! When the dimension is
  1, it's obvious. Assume the Theorem holds
  if the dimension less than $n$. Now,
  if $f = 0$, it's also obvious because it's
  diagonal form under every basis. If
  $f \neq 0$, there must be some vector $\ee_1$
  such that $q(\ee_1) \neq 0$, otherwise for
  all $\xx, \yy, f(\xx, \yy) = 0$ based on
  definition. $f(\ee_1, \ee_1) = q(\ee_1)
  \neq 0$.

  Consider the linear function $f_1: \xx
  \rightarrow f(\xx, \ee_1)$. We know
  $\dim \ker f_1 = n - 1$ because it's linear
  function. Restrict $f$ on $\ker f_1$. Using
  the assumption, we have a canonical basis
  $\langle \ee_2, \dots, \ee_n \rangle$.
  Because they span the kernel space,
  $f(\ee_i , \ee_1) = 0, \forall i \neq 1$.
  Thus, $f(\ee_i, \ee_j) = 0, \forall i\neq j$.
  \Nie{For non-symmetric matrix, we cannot
  conclude the last statement.}
  
  Finally, we should verify they are linearly
  independent. By contradiction, $\ee_1$ is
  linear combination of $\ee_2, \dots, \ee_n$,
  however, $f$ is bilinear which means
  $f(\ee_1, \ee_1) = 0$.
\end{Proof}

Based on above theorems, we can calculate the
quadratic form under the canonical basis,
namely $q(\xx) = \lambda_1 x_1^2 + \dots +
\lambda_n x_n^2$. Unfortunately, $\lambda_i$ is
not an invariant independent of the choice of
basis vectors. For example, we swap the $\ee_1$
and $\ee_2$ and $\lambda_1' = \lambda_2,
\lambda_2' = \lambda_1$. For another example,
if $\RR$ is real number field, we can scale the basis $\sqrt{\lambda_i}\ee_i$
if $\lambda_i \geq 0$ and $\sqrt{-\lambda_i}
\ee_i$ if $\lambda_i < 0$. In this case,
$q(\xx) = x_1^2 + \dots + x_s^2 - x_{s+1}^2 - \dots -
x_{r}^2$ where $r$ is the rank of $f$ and a
basis-independent invariant. Naturally, we are
wondering $s$ is an invariant or not.

\begin{Theorem}[Sylvester's Law of Inertia]
  As above, $s$ is a basis-independent
  invariant.
\end{Theorem}

\begin{Proof}
  Suppose $q(\xx) = (x_1')^2 + \dots + (x_t')^2
  - (x_{t+1}')^2 - \dots - (x_r')^2$ under some
  basis s.t. $t \neq s$. Without loss of
  generality, we suppose $t < s$. Consider two
  space $L_s = \langle \ee_1, \dots, \ee_s
  \rangle$ and $L_t = \langle \ee_{t+1}',
  \dots, \ee_{n}' \rangle$. We claim $L_s \cap
  L_t \neq 0$ because $\dim{L_s \cap L_t} =
  \dim{L_s} + \dim{L_t} - \dim{\{ L_s + L_t \}}
  \geq s + n - t - n > 0$. There exists a
  $\xx = \sum_{i=1}^s x_i \ee_i =
  \sum_{j = t+1}^n x_j' \ee_j' \neq 0$.
  Then $q(\xx)= q(\sum_{i=1}^s x_i \ee_i) > 0$
  and $q(\xx) = q(\sum_{j = t+1}^n x_j' \ee_j')
  \leq 0$ lead to a contradiction.
\end{Proof}

\begin{Define}
  A non-degenerate quadratic form \( q: V \to \mathbb{R} \) is called \textbf{positive definite} (or \textbf{negative definite}) if
\[
q(x) > 0 \quad (\text{or } q(x) < 0)
\]
for all nonzero vectors \( x \ne 0 \).

The quadratic form \( q \) is called \textbf{positive semidefinite} (or \textbf{negative semidefinite}) if
\[
q(x) \ge 0
\]
for all \( x \in V \).

Finally, the quadratic form \( q \) is called \textbf{indefinite} if it takes both positive and negative values.
\end{Define}

\begin{Theorem}
  The quadratic form $q$ corresponds to a
  symmetric bilinear form $f$ and a symmetric
  matrix $F$. Equivalently, $q$ is positive
  semidefine if the eigenvalue
  $\lambda_i(F)$ satisfy $\forall i \in [n],
  \lambda_i(F) \geq 0$ and is positive
  semidefine if the eigenvalue $\lambda_i(F)$
  satisfy $\forall i \in [n], \lambda_i(F) >0$.
\end{Theorem}

Thanks to my Linear Algebra contextbook.
\begin{Proof}
  First of all, we know the symmetric bilinear
  form is diagonalizable. According to
  theorem~\ref{thm:Diagonal}, it implies two
  facts. One is that all eigenvalues are real
  number and it's valid to compare to zero.
  The other is that all eigenvectors can span
  the whole space.

  In other words, for all $\xx \in V$, there
  exists a $\lambda$ s.t. $F\xx = \lambda\xx$.
  Then $q(\xx) = \xx^\top F \xx =
  \lambda \xx^\top \xx$, which means $q(\xx)
  \geq 0$ if and only if $\lambda \geq 0$.
  Together with the second fact, $\forall \xx.
  q(\xx) \geq 0$ if and only if $\forall i \in
  [n]. \lambda_i \geq 0$.
\end{Proof}

\subsubsection{Eigenvector and Eigenvalue}

\begin{Theorem}
  Eigenvectors corresponding to different
  eigenvalues must be linearly independent.
  The sum of eigenvectors, namely
  $\sum_{\lambda_i \in \Spec{\opA}} V^\lambda$ is
  direct sum.
  Generally, it's not the original space.
\end{Theorem}

\begin{Proof}
  Prove it by induction. Suppoe we select
  $m$ different eigenvalues.
  If $m=1$, it holds.
  Assume our statement holds for each $m < n$.

  If it is not true when $m = n$, then select
  some linearly dependent vectors $\ee_1,\dots,
  \ee_m$. We have,
  $$
  \alpha_1 \ee_1 + \dots + \alpha_m \ee_m=0,
  $$
  where $\alpha_i$ is non-trivial.

  Apple the operator $\opA$ to both sides of
  the equation. We have
  $$
  \lambda_1 \alpha_1 \ee_1 + \dots + \lambda_m
  \alpha_m \ee_m = 0.
  $$

  Subtract them:

  $$
  (\lambda_1 - \lambda_m)\alpha_1 \ee_1 +
  (\lambda_2 - \lambda_m)\alpha_2 \ee_2 + \dots
  + (\lambda_{m-1} - \lambda_m)\alpha_{m-1}
  \ee_{m-1} = 0.
  $$
  It leads to a contradiction that if we
  select $m-1$ different eigenvalues, they are
  linearly dependent.
\end{Proof}


\begin{Theorem}[Diagonalizable]
  \label{thm:Diagonal}
  Suppose \( A \) is a linear operator on a finite-dimensional vector space \( V \) over the field \( \RR \).  
  A necessary and sufficient condition for \(
  \opA \) to be diagonalizable is that the following two conditions are satisfied:

\begin{enumerate}
    \item All roots of the characteristic polynomial \( \chi_A(t) \) are in \( \RR \);
    \item The geometric multiplicity of each eigenvalue \( \lambda \) equals its algebraic multiplicity.
\end{enumerate}
\end{Theorem}

\begin{Proof}
  $\Rightarrow:$ Suppose two conditions hold.
  Let $\lambda_1, \dots, \lambd_m$ are the
  roots of the characteristic polynomial
  $\chi_A(t)$, while $k_1, \dots, k_m$ are
  their multiplicity. We know $\dim{
    V^{\lambda_1} + \dots + V^{\lambda_m}} =
  \dim{V^{\lambda_1}} + \dots +
  \dim{V^{\lambda_m}}$, therefore, $V =
  V^{\lambda_1} \oplus \dots\oplus V^{\lambda_m}$.
  All the basis of $V^{\lambda_1}, \dots,
  V^{\lambda_m}$ consists the basis of $V$.
  It's obvious that $\opA$ is diagonal under
  such basis.

  $\Leftarrow:$ I omit details because it's
  more straightforward.  
\end{Proof}


\subsection{SDP Formulation}
\subsubsection{Spectrahedron}

\begin{Define}[LMI]
  A linear matrix inequality (LMI) has the
  form
  \[
    A_0 + \sum_{i=1}^m A_ix_i \succeq 0,
  \]
  where $A_i \in \mathcal{S}^n$ are given
  symmetric matrices.
\end{Define}

\begin{Define}[Spectrahedron]
  A set $S \subset \RR^m$ is a spectrahedron if
  it has the form
  \[
    S = \{(x_1,\dots,x_m) \in \RR^m :
    A_0 + \sum_{i=1}^m A_ix_i \succeq 0\},
  \]
  for some given symmetric matrices $A_i \in
  \mathcal{S}^n$.
\end{Define}

\subsubsection{Primal SDP Formulation}
\begin{Define}
  \normalfont
  An SDP problem in standard primal form is written as
  \begin{align*}
    \text{minimize} \quad & \langle C,X \rangle \\
    \text{subject to} \quad &\langle A_i, X \rangle =b_i,
                              \quad i = 1, \dots, m\\
                          & X \succeq 0.
  \end{align*}
  where $C, A_i \in \mathcal{S}^n$, and $\langle X,Y
  \rangle := \tr(X^\top Y) = \sum_{ij}X_{ij}Y_{ij}$.

\end{Define}

\subsubsection{Dual SDP Formulation}
\begin{Define}
  \normalfont
  An SDP problem in dual form is written as
  \begin{align*}
    \text{maximize} \quad & b^\top y \\
    \text{subject to} \quad & \sum_{i=1}^m A_iy_i
                              \preceq C,
  \end{align*}
  where $b = (b_i,\dots b_m)$, and $y = (y_1,\dots, y_m)$
  are the dual decision variables.
\end{Define}


\end{document}
